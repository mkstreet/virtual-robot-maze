\documentclass[12pt]{article}
\usepackage{geometry}
\usepackage{hyperref}
\usepackage{listings}
\usepackage{xcolor}

\geometry{margin=1in}

% Define Rust code listing style
\lstdefinelanguage{Rust}{
	keywords={fn, let, mut, println},
	keywordstyle=\color{blue}\bfseries,
	ndkeywords={i32, String, bool},
	ndkeywordstyle=\color{purple}\bfseries,
	comment=[l]{//},
	commentstyle=\color{gray}\ttfamily,
	stringstyle=\color{red}\ttfamily,
	sensitive=true
}

\lstset{
	language=Rust,
	basicstyle=\ttfamily\small,
	columns=flexible,
	numbers=left,
	numberstyle=\tiny\color{gray},
	stepnumber=1,
	numbersep=5pt,
	showspaces=false,
	showstringspaces=false,
	tabsize=4,
	breaklines=true,
	breakatwhitespace=true,
	frame=single,
	backgroundcolor=\color{white},
	captionpos=b
}

\title{Stepping-Stone Task: Basic Syntax and Variables}
\author{}
\date{}

\begin{document}
	
	\maketitle
	
	\section*{Objective}
	The goal of this task is to introduce you to basic programming concepts in Rust, focusing on variables, data types, and basic operations.
	
	\section*{What are Variables?}
	A \textbf{variable} is like a container that holds a value. You can think of it as a labeled box where you can store information. For example, you can have a box labeled \texttt{age} that holds the number 15. You can use variables to store numbers, text, and other types of information so that you can use or change them later in your program.
	
	\section*{What Does "Declare" Mean?}
	To \textbf{declare} a variable means to create it and give it a name. Declaring a variable also tells the computer what kind of value it will hold (e.g., a number, a word, or true/false). You declare a variable so that you can use it in your program. Think of it as setting up a new box to store a specific type of information.
	
	\section*{What are Data Types?}
	In programming, \textbf{data types} define what kind of information a variable can hold. Here are some basic data types in Rust:
	\begin{itemize}
		\item \textbf{Integer (\texttt{i32})}: A whole number, like 5 or 10. The \texttt{i32} type means it's a 32-bit integer, which can hold positive or negative whole numbers.
		\item \textbf{Boolean (\texttt{bool})}: A value that can be either true or false, used for yes/no or on/off situations.
		\item \textbf{String (\texttt{\&str})}: A sequence of characters, like "Hello, Rust!" or "123 Main St". In Rust, strings are represented as \texttt{\&str}, which is a reference to text.
	\end{itemize}
	
	\section*{Task Overview}
	You will write a simple Rust program that:
	\begin{itemize}
		\item Declares variables of different data types (e.g., integers, strings, booleans).
		\item Performs basic arithmetic operations.
		\item Prints the results to the screen.
	\end{itemize}
	
	\section*{Instructions for Codespaces}
	\begin{enumerate}
		\item \textbf{Open Your Project in GitHub Codespaces}
		\begin{itemize}
			\item Go to your forked \texttt{virtual-robot-maze} repository on GitHub.
			\item Click the \texttt{Code} button and select \texttt{Codespaces}.
			\item Create a new codespace, or open an existing one.
		\end{itemize}
		
		\item \textbf{Navigate to the \texttt{basic-programming} directory}
		\begin{itemize}
			\item In Codespaces, open the terminal (usually at the bottom of the screen).
			\item Navigate to the \texttt{basic-programming} folder that already exists in the project:
			\begin{lstlisting}[language=bash]
				cd basic-programming
			\end{lstlisting}
		\end{itemize}
		
		\item \textbf{Create a new Rust project for this task}
		\begin{itemize}
			\item Inside the \texttt{basic-programming} directory, create a new Rust project:
			\begin{lstlisting}[language=bash]
				cargo new basic_syntax
				cd basic_syntax
			\end{lstlisting}
			\item This will create a folder named \texttt{basic\_syntax} where you will work on this task.
		\end{itemize}
		
		\item \textbf{Declare Variables}
		\begin{itemize}
			\item Declare variables of different types, like integers, strings, and booleans.
			\item Use the \texttt{let} keyword to define variables:
			\begin{lstlisting}[language=Rust]
				let x: i32 = 5;            // Integer
				let name: &str = "Rust";   // String
				let is_learning: bool = true; // Boolean
			\end{lstlisting}
			\item Modify the values of these variables to understand how they work.
		\end{itemize}
		
		\item \textbf{Perform Basic Arithmetic}
		\begin{itemize}
			\item Add two integer variables together and store the result in a new variable:
			\begin{lstlisting}[language=Rust]
				let y: i32 = 10;
				let sum = x + y;
				println!("The sum is: {}", sum);
			\end{lstlisting}
			\item Try other operations like subtraction, multiplication, and division.
		\end{itemize}
		
		\item \textbf{Print Results to the Screen}
		\begin{itemize}
			\item Use the \texttt{println!} macro to display the values of the variables and results of operations:
			\begin{lstlisting}[language=Rust]
				println!("Name: {}", name);
				println!("Is learning Rust: {}", is_learning);
			\end{lstlisting}
		\end{itemize}
	\end{enumerate}
	
	\section*{What to Expect}
	\begin{itemize}
		\item After running the program, you should see output similar to:
		\begin{verbatim}
			The sum is: 15
			Name: Rust
			Is learning Rust: true
		\end{verbatim}
		\item Modify the values of variables and re-run the program to observe different results.
	\end{itemize}
	
	\section*{Testing and Debugging}
	\begin{itemize}
		\item To run your program, use:
		\begin{lstlisting}[language=bash]
			cargo run
		\end{lstlisting}
		\item If you see an error message, try to understand what it says—Rust’s error messages are often helpful.
		\item If the program runs successfully, try adding new variables or operations.
	\end{itemize}
	
	\section*{Hints}
	\begin{itemize}
		\item Variables must have the correct data types, so \texttt{i32}, \texttt{\&str}, and \texttt{bool} must match the values you assign.
		\item Experiment with different arithmetic operations (e.g., \texttt{-}, \texttt{*}, \texttt{/}) and see what results you get.
		\item If you get stuck, break the problem into smaller steps—start with one variable and build up from there.
	\end{itemize}
	
	\section*{Next Steps}
	Once you’ve completed this task, you will have a better understanding of how variables and data types work in Rust. This foundation will be helpful for the upcoming tasks in the virtual robot project.
	
\end{document}
