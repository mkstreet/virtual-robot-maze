\documentclass{article}
\usepackage{geometry}
\usepackage{listings}
\usepackage{xcolor}

\geometry{margin=0.5in}

\title{Programming 101: Introduction to Programming Concepts with Rust}
\author{}
\date{}

% Define Rust code listing style
\lstdefinelanguage{Rust}{
	keywords={fn, pub, let, mut, return, struct, impl, for, in, if, else, match, while, loop},
	keywordstyle=\color{blue}\bfseries,
	ndkeywords={u8, i32, f64, String, bool, Option, Result, Some, None, Ok, Err},
	ndkeywordstyle=\color{purple}\bfseries,
	comment=[l]{//},
	commentstyle=\color{gray}\ttfamily,
	stringstyle=\color{red}\ttfamily,
	sensitive=true
}

\lstset{
	language=Rust,
	basicstyle=\ttfamily\small,
	columns=flexible,
	numbers=left,
	numberstyle=\tiny\color{gray},
	stepnumber=1,
	numbersep=5pt,
	showspaces=false,
	showstringspaces=false,
	tabsize=4,
	breaklines=true,
	breakatwhitespace=true,
	frame=single,
	backgroundcolor=\color{white},
	captionpos=b
}

\begin{document}
	
	\maketitle
	
	\section*{Introduction}
	Welcome to \textbf{Programming 101}! This guide introduces basic programming concepts using the Rust language. Whether you’re new to programming or just learning Rust, this document will help you understand fundamental ideas and provide code examples to build your skills.
	
	\section*{Basic Concepts}
	\subsection*{Variables and Data Types}
	Variables store values that can be used in a program. In Rust, variables are declared using the \texttt{let} keyword. Rust supports various data types like integers, floating-point numbers, booleans, and strings.
	
	\begin{lstlisting}[language=Rust, caption={Variable declaration and data types in Rust}]
		fn main() {
			// Immutable variable
			let x = 5; // i32 (integer)
			
			// Mutable variable
			let mut y = 10; // i32 (integer)
			y = 15; // Now y is 15
			
			// Float type
			let pi: f64 = 3.14;
			
			// Boolean type
			let is_rust_fun = true;
			
			// String type
			let greeting = String::from("Hello, Rust!");
			
			println!("x: {}, y: {}, pi: {}, is_rust_fun: {}, greeting: {}", x, y, pi, is_rust_fun, greeting);
		}
	\end{lstlisting}
	\newpage
	\subsection*{Control Flow}
	Control flow allows you to make decisions in your code using \texttt{if}, \texttt{else}, and loops like \texttt{while} and \texttt{for}.
	
	\begin{lstlisting}[language=Rust, caption={If statements and loops}]
		fn main() {
			let age = 18;
			
			// If-Else statement
			if age >= 18 {
				println!("You are an adult.");
			} else {
				println!("You are a minor.");
			}
			
			// For loop
			for i in 0..5 {
				println!("Loop iteration: {}", i);
			}
			
			// While loop
			let mut count = 0;
			while count < 3 {
				println!("Count: {}", count);
				count += 1;
			}
		}
	\end{lstlisting}
	
	\subsection*{Functions}
	Functions are reusable blocks of code that perform a specific task. They are declared using the \texttt{fn} keyword in Rust.
	
	\begin{lstlisting}[language=Rust, caption={Functions in Rust}]
		fn main() {
			greet("Rust");
			let sum = add(5, 10);
			println!("Sum: {}", sum);
		}
		
		// Function to print a greeting
		fn greet(name: &str) {
			println!("Hello, {}!", name);
		}
		
		// Function to add two numbers
		fn add(a: i32, b: i32) -> i32 {
			a + b // Implicit return
		}
	\end{lstlisting}
	\newpage
	\section*{Advanced Concepts in Rust}
	Rust introduces unique features such as \textbf{ownership}, \textbf{borrowing}, and \textbf{lifetimes}, which help ensure memory safety.
	
	\subsection*{Ownership}
	In Rust, each value has a single owner. When ownership is transferred, the original owner can no longer use the value.
	
	\begin{lstlisting}[language=Rust, caption={Ownership in Rust}]
		fn main() {
			let s1 = String::from("hello");
			let s2 = s1; // Ownership is moved to s2
			
			// println!("{}", s1); // Error: s1 is no longer valid
			println!("{}", s2); // s2 is valid
		}
	\end{lstlisting}
	
	\subsection*{Borrowing}
	Borrowing allows a value to be accessed without transferring ownership, using references.
	
	\begin{lstlisting}[language=Rust, caption={Borrowing in Rust}]
		fn main() {
			let s1 = String::from("hello");
			let len = calculate_length(&s1); // Borrow s1
			println!("Length of '{}': {}", s1, len); // s1 is still valid
		}
		
		fn calculate_length(s: &String) -> usize {
			s.len() // Use the reference
		}
	\end{lstlisting}
	
	\subsection*{Lifetimes}
	Lifetimes ensure that references are valid as long as needed. They prevent dangling references.
	
	\begin{lstlisting}[language=Rust, caption={Lifetimes in Rust}]
		fn longest<'a>(x: &'a str, y: &'a str) -> &'a str {
			if x.len() > y.len() {
				x
			} else {
				y
			}
		}
		
		fn main() {
			let string1 = String::from("Rust");
			let string2 = String::from("Programming");
			let result = longest(&string1, &string2);
			println!("The longest string is '{}'", result);
		}
	\end{lstlisting}
	
	\section*{Conclusion}
	This guide has introduced basic programming concepts, control flow, and Rust-specific features like ownership and borrowing. These foundations will help you as you start coding the virtual robot to navigate the maze!
	
\end{document}
