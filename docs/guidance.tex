\documentclass[12pt]{article}
\usepackage{geometry}
\usepackage{hyperref}
\usepackage{listings}
\usepackage{xcolor}

\geometry{margin=1in}

% Define Rust code listing style
\lstdefinelanguage{Rust}{
	keywords={fn, let, mut, pub, use, mod, struct, enum, impl, for, while, loop, if, else, match, return, break, continue},
	keywordstyle=\color{blue}\bfseries,
	ndkeywords={u8, i32, bool, String, Result, Option},
	ndkeywordstyle=\color{purple}\bfseries,
	comment=[l]{//},
	commentstyle=\color{gray}\ttfamily,
	stringstyle=\color{red}\ttfamily,
	sensitive=true
}

\lstset{
	language=Rust,
	basicstyle=\ttfamily\small,
	columns=flexible,
	numbers=left,
	numberstyle=\tiny\color{gray},
	stepnumber=1,
	numbersep=5pt,
	showspaces=false,
	showstringspaces=false,
	tabsize=4,
	breaklines=true,
	breakatwhitespace=true,
	frame=single,
	backgroundcolor=\color{white},
	captionpos=b
}

\title{Guidance for Students}
\author{}
\date{}

\begin{document}
	
	\maketitle
	
	\section*{Introduction}
	Welcome to the \texttt{virtual-robot-maze} project! In this repository, you will work on various tasks, ranging from basic programming exercises to controlling a virtual robot. This document provides an overview of how to navigate the project, use GitHub Codespaces, and work with multiple Rust projects.
	
	\section*{Getting Started with GitHub and Codespaces}
	\begin{enumerate}
		\item \textbf{Fork the Repository}
		\begin{itemize}
			\item Go to the main repository on GitHub.
			\item Click the \texttt{Fork} button at the top right to create your own copy of the project.
			\item You can find the forked repository under your GitHub profile.
		\end{itemize}
		
		\item \textbf{Open GitHub Codespaces}
		\begin{itemize}
			\item In your forked repository, click the \texttt{Code} button and select \texttt{Codespaces}.
			\item If you don’t have a codespace yet, create one by clicking \texttt{New Codespace}.
			\item Once it loads, you’ll have a development environment with Rust pre-installed, ready for coding.
		\end{itemize}
	\end{enumerate}
	
	\section*{Working with Multiple Rust Projects}
	The \texttt{virtual-robot-maze} repository contains multiple Rust projects, each focused on different tasks:
	\begin{itemize}
		\item \textbf{Basic Programming Tasks}: These are located in the \texttt{basic-programming} directory and are designed to teach fundamental Rust programming concepts.
		\item \textbf{Virtual Robot Tasks}: These tasks are found in the \texttt{src} directory and involve controlling a virtual robot using Rust.
	\end{itemize}
	
	\subsection*{Switching Between Projects}
	As you work on different tasks, you'll need to switch between Rust projects. Here’s how:
	\begin{enumerate}
		\item \textbf{To switch to the basic programming tasks}:
		\begin{itemize}
			\item Navigate to the \texttt{basic-programming} directory:
			\begin{lstlisting}[language=bash]
				cd ~/workspaces/virtual-robot-maze/basic-programming/basic_syntax
			\end{lstlisting}
			\item Use \texttt{cargo run} to run the code in the basic programming project.
		\end{itemize}
		
		\item \textbf{To switch to the virtual robot tasks}:
		\begin{itemize}
			\item Navigate to the \texttt{src} directory:
			\begin{lstlisting}[language=bash]
				cd ~/workspaces/virtual-robot-maze/src
			\end{lstlisting}
			\item Use \texttt{cargo run} to execute the virtual robot code.
		\end{itemize}
	\end{enumerate}
	
	\section*{General Workflow Tips}
	\begin{itemize}
		\item Always use the terminal to navigate between project directories before running commands.
		\item Use \texttt{pwd} (print working directory) to check which directory you are currently in.
		\item Save your changes frequently and use \texttt{git commit} and \texttt{git push} to save progress to your GitHub repository.
	\end{itemize}
	
	\section*{Next Steps}
	Once you understand how to navigate the project and switch between tasks, you can start working on the stepping-stone tasks or proceed directly to the virtual robot tasks. Refer to specific task documents for detailed instructions.
	
\end{document}
